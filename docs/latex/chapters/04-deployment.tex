\chapter{Deployment}

\section{Prerequisites}
Before deploying Cascade, ensure you have the following tools installed:
\begin{itemize}
    \item Docker and Docker Compose
    \item Google Cloud SDK (gcloud)
    \item kubectl
    \item Helm
\end{itemize}

\section{Local Development}
For local development, Docker Compose is used to spin up the entire stack.
\begin{lstlisting}[language=bash, caption=Start Local Environment]
docker-compose up --build
\end{lstlisting}
This starts all microservices, MongoDB, Redis, and Kafka locally.

\section{GKE Deployment}
Deploying to Google Kubernetes Engine involves several steps, automated via Helm and shell scripts.

\subsection{Setup Script}
The \texttt{setup.sh} script automates the cluster creation and configuration.
\begin{lstlisting}[language=bash, caption=Run Setup Script]
./setup.sh <YOUR_PROJECT_ID>
\end{lstlisting}

\subsection{Helm Charts}
The project uses a unified Helm chart located in \texttt{helm/cascade}.
Key configuration files:
\begin{itemize}
    \item \texttt{values.yaml}: Default configuration values.
    \item \texttt{templates/}: Kubernetes manifest templates.
\end{itemize}

\subsection{Manual Deployment Steps}
If not using the setup script:
\begin{enumerate}
    \item \textbf{Create Cluster}: Create a GKE cluster with standard settings.
    \item \textbf{Install Dependencies}: Install Nginx Ingress Controller and Cert Manager (if needed).
    \item \textbf{Build Images}: Build and push Docker images to Google Artifact Registry.
    \item \textbf{Deploy Helm Chart}:
    \begin{lstlisting}[language=bash]
    helm install cascade ./helm/cascade --set project.id=<PROJECT_ID>
    \end{lstlisting}
\end{enumerate}

\section{Configuration}
\subsection{Environment Variables}
Services are configured via environment variables injected by Kubernetes.
\begin{itemize}
    \item \texttt{MONGODB\_URI}: Connection string for MongoDB.
    \item \texttt{KAFKA\_BROKERS}: Address of Kafka bootstrap server.
    \item \texttt{REDIS\_HOST}: Address of Redis instance.
    \item \texttt{JWT\_SECRET}: Secret key for signing tokens.
\end{itemize}

\subsection{Secrets Management}
Sensitive data like OAuth secrets and database credentials should be managed using Kubernetes Secrets, not plain text in \texttt{values.yaml}.
